\documentclass[]{article}
\usepackage{lmodern}
\usepackage{setspace}
\setstretch{1}
\usepackage{amssymb,amsmath}
\usepackage{ifxetex,ifluatex}
\usepackage{fixltx2e} % provides \textsubscript
\ifnum 0\ifxetex 1\fi\ifluatex 1\fi=0 % if pdftex
  \usepackage[T1]{fontenc}
  \usepackage[utf8]{inputenc}
\else % if luatex or xelatex
  \ifxetex
    \usepackage{mathspec}
  \else
    \usepackage{fontspec}
  \fi
  \defaultfontfeatures{Ligatures=TeX,Scale=MatchLowercase}
\fi
% use upquote if available, for straight quotes in verbatim environments
\IfFileExists{upquote.sty}{\usepackage{upquote}}{}
% use microtype if available
\IfFileExists{microtype.sty}{%
\usepackage{microtype}
\UseMicrotypeSet[protrusion]{basicmath} % disable protrusion for tt fonts
}{}
\usepackage[margin=1in]{geometry}
\usepackage{hyperref}
\PassOptionsToPackage{usenames,dvipsnames}{color} % color is loaded by hyperref
\hypersetup{unicode=true,
            pdftitle={Component response rate variation underlies the stability of complex systems (revision notes)},
            pdfauthor={A. Bradley Duthie ( alexander.duthie@stir.ac.uk )},
            colorlinks=true,
            linkcolor=blue,
            citecolor=Blue,
            urlcolor=Blue,
            breaklinks=true}
\urlstyle{same}  % don't use monospace font for urls
\usepackage{graphicx,grffile}
\makeatletter
\def\maxwidth{\ifdim\Gin@nat@width>\linewidth\linewidth\else\Gin@nat@width\fi}
\def\maxheight{\ifdim\Gin@nat@height>\textheight\textheight\else\Gin@nat@height\fi}
\makeatother
% Scale images if necessary, so that they will not overflow the page
% margins by default, and it is still possible to overwrite the defaults
% using explicit options in \includegraphics[width, height, ...]{}
\setkeys{Gin}{width=\maxwidth,height=\maxheight,keepaspectratio}
\IfFileExists{parskip.sty}{%
\usepackage{parskip}
}{% else
\setlength{\parindent}{0pt}
\setlength{\parskip}{6pt plus 2pt minus 1pt}
}
\setlength{\emergencystretch}{3em}  % prevent overfull lines
\providecommand{\tightlist}{%
  \setlength{\itemsep}{0pt}\setlength{\parskip}{0pt}}
\setcounter{secnumdepth}{0}
% Redefines (sub)paragraphs to behave more like sections
\ifx\paragraph\undefined\else
\let\oldparagraph\paragraph
\renewcommand{\paragraph}[1]{\oldparagraph{#1}\mbox{}}
\fi
\ifx\subparagraph\undefined\else
\let\oldsubparagraph\subparagraph
\renewcommand{\subparagraph}[1]{\oldsubparagraph{#1}\mbox{}}
\fi

%%% Use protect on footnotes to avoid problems with footnotes in titles
\let\rmarkdownfootnote\footnote%
\def\footnote{\protect\rmarkdownfootnote}

%%% Change title format to be more compact
\usepackage{titling}

% Create subtitle command for use in maketitle
\providecommand{\subtitle}[1]{
  \posttitle{
    \begin{center}\large#1\end{center}
    }
}

\setlength{\droptitle}{-2em}

  \title{Component response rate variation underlies the stability of complex
systems (revision notes)}
    \pretitle{\vspace{\droptitle}\centering\huge}
  \posttitle{\par}
    \author{A. Bradley Duthie (
\href{mailto:alexander.duthie@stir.ac.uk}{\nolinkurl{alexander.duthie@stir.ac.uk}}
)}
    \preauthor{\centering\large\emph}
  \postauthor{\par}
      \predate{\centering\large\emph}
  \postdate{\par}
    \date{Biological and Environmental Sciences, University of Stirling, Stirling,
UK, FK9 4LA}

\usepackage{amsmath}
\usepackage{natbib}
\usepackage{lineno}
\usepackage[utf8]{inputenc}
\linenumbers
\bibliographystyle{amnatnat}

\begin{document}
\maketitle

\hypertarget{role-of-correlated-matrices-in-stabilisation}{%
\section{Role of correlated matrices in
stabilisation}\label{role-of-correlated-matrices-in-stabilisation}}

In complex systems represented by large random matrices, correlation
between matrix elements \(A_{ij}\) and \(A_{ji}\) affects the
distribution of eigenvalues and therefore local stability. As the
correlation between matrix elements (\(\rho\)) decreases, the eigenvalue
spectra changes such that more variation falls along the imaginary axis.
The figure panels below compare a random matrix in which \(\rho = 0\)
(left) to one in which (\(\rho\) = -0.5). In both cases, complex systems
include \(S = 1000\) components, with diagonal elements of \(-1\) and
off-diagonal elements drawn from a normal distribution with a mean of
\(0\) and \(\sigma = 0.4\).

\includegraphics{revision_notes_files/figure-latex/unnamed-chunk-2-1.pdf}

Because of this affect of \(\rho\) on the eigenvalue spectra, decreasing
values of \(\rho\) will also decrease the rightmost eigenvalue of the
matrix \(A\). This makes it more likely that the complex system
represented by \(A\) is locally stable, as stability occurs when all
real parts of eigenvalues are negative. Note that this elongation along
the imaginary axis is also characteristic of predator-prey communities
(in which, by definition \(A_{ij}\) and \(A_{ji}\) have opposing signs).
Also note that as \(\rho\) increases such that \(\rho > 0\), the same
elongation happens along the real axis, making random complex systems
less likely to be stable.

A simple numerical analysis illustrates the linear relationship between
\(\rho\) and the expected value of the real part of the leading
eigenvalue \(\max(\Re(\lambda))\). Below, I have run 1000 simulations
across values of \(\rho\) from -0.9 to 0.9 for complex systems with
\(S = 25\) components.

\includegraphics{revision_notes_files/figure-latex/unnamed-chunk-4-1.pdf}

Error bars show 95\% bootstrapped confidence intervals.


\end{document}
