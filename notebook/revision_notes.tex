\documentclass[]{article}
\usepackage{lmodern}
\usepackage{setspace}
\setstretch{1}
\usepackage{amssymb,amsmath}
\usepackage{ifxetex,ifluatex}
\usepackage{fixltx2e} % provides \textsubscript
\ifnum 0\ifxetex 1\fi\ifluatex 1\fi=0 % if pdftex
  \usepackage[T1]{fontenc}
  \usepackage[utf8]{inputenc}
\else % if luatex or xelatex
  \ifxetex
    \usepackage{mathspec}
  \else
    \usepackage{fontspec}
  \fi
  \defaultfontfeatures{Ligatures=TeX,Scale=MatchLowercase}
\fi
% use upquote if available, for straight quotes in verbatim environments
\IfFileExists{upquote.sty}{\usepackage{upquote}}{}
% use microtype if available
\IfFileExists{microtype.sty}{%
\usepackage{microtype}
\UseMicrotypeSet[protrusion]{basicmath} % disable protrusion for tt fonts
}{}
\usepackage[margin=1in]{geometry}
\usepackage{hyperref}
\PassOptionsToPackage{usenames,dvipsnames}{color} % color is loaded by hyperref
\hypersetup{unicode=true,
            pdftitle={Component response rate variation underlies the stability of complex systems (revision notes)},
            pdfauthor={A. Bradley Duthie ( alexander.duthie@stir.ac.uk )},
            colorlinks=true,
            linkcolor=blue,
            citecolor=Blue,
            urlcolor=Blue,
            breaklinks=true}
\urlstyle{same}  % don't use monospace font for urls
\usepackage{graphicx,grffile}
\makeatletter
\def\maxwidth{\ifdim\Gin@nat@width>\linewidth\linewidth\else\Gin@nat@width\fi}
\def\maxheight{\ifdim\Gin@nat@height>\textheight\textheight\else\Gin@nat@height\fi}
\makeatother
% Scale images if necessary, so that they will not overflow the page
% margins by default, and it is still possible to overwrite the defaults
% using explicit options in \includegraphics[width, height, ...]{}
\setkeys{Gin}{width=\maxwidth,height=\maxheight,keepaspectratio}
\IfFileExists{parskip.sty}{%
\usepackage{parskip}
}{% else
\setlength{\parindent}{0pt}
\setlength{\parskip}{6pt plus 2pt minus 1pt}
}
\setlength{\emergencystretch}{3em}  % prevent overfull lines
\providecommand{\tightlist}{%
  \setlength{\itemsep}{0pt}\setlength{\parskip}{0pt}}
\setcounter{secnumdepth}{0}
% Redefines (sub)paragraphs to behave more like sections
\ifx\paragraph\undefined\else
\let\oldparagraph\paragraph
\renewcommand{\paragraph}[1]{\oldparagraph{#1}\mbox{}}
\fi
\ifx\subparagraph\undefined\else
\let\oldsubparagraph\subparagraph
\renewcommand{\subparagraph}[1]{\oldsubparagraph{#1}\mbox{}}
\fi

%%% Use protect on footnotes to avoid problems with footnotes in titles
\let\rmarkdownfootnote\footnote%
\def\footnote{\protect\rmarkdownfootnote}

%%% Change title format to be more compact
\usepackage{titling}

% Create subtitle command for use in maketitle
\providecommand{\subtitle}[1]{
  \posttitle{
    \begin{center}\large#1\end{center}
    }
}

\setlength{\droptitle}{-2em}

  \title{Component response rate variation underlies the stability of complex
systems (revision notes)}
    \pretitle{\vspace{\droptitle}\centering\huge}
  \posttitle{\par}
    \author{A. Bradley Duthie (
\href{mailto:alexander.duthie@stir.ac.uk}{\nolinkurl{alexander.duthie@stir.ac.uk}}
)}
    \preauthor{\centering\large\emph}
  \postauthor{\par}
      \predate{\centering\large\emph}
  \postdate{\par}
    \date{Biological and Environmental Sciences, University of Stirling, Stirling,
UK, FK9 4LA}

\usepackage{amsmath}
\usepackage{natbib}
\usepackage{lineno}
\usepackage[utf8]{inputenc}
\linenumbers
\bibliographystyle{amnatnat}

\begin{document}
\maketitle

\hypertarget{effects-of-correlations-interactions}{%
\section{Effects of correlations
interactions}\label{effects-of-correlations-interactions}}

If the off-diagonal elements of \(\textbf{A}\) are sampled independently
from an identical distribution, then \(\max(\Re(\lambda))\) for
\(\textbf{M} = \textbf{A}\) can be estimated from five
values\textsuperscript{\protect\hyperlink{ref-Tang2014b}{1}}. These
values include (1) system size (\(S\)), (2) mean self-regulation of
components (\(d\)), (3) mean interaction strength between components
(\(\mu\)), (4) the standard deviation between component interaction
strengths (\(\sigma\)), and (5) the correlation of interaction strengths
between components, \(M_{ij}\) and \(M_{ji}\) (\(\rho\)). When
investigating the effect of varying component response rate
\(Var(\gamma)\) on stability by defining
\(\textbf{M} = \gamma\textbf{A}\), \(S\) remains unchanged. Further,
values of \(\gamma_{i}\) were sampled such that \(E[d]\) and \(E[\mu]\)
also remained unchanged (in practice, diagonal elements of
\(\textbf{M}\) were standardised so that mean values were identical
before and after adding \(\gamma\)). What \(Var(\gamma)\) does change is
the variation in component interaction strengths, and \(\rho\).

Variation in \(\gamma\) increases the total variation such that for
\(\textbf{M} = \gamma \textbf{A}\),
\(\sigma^{2}_{M_{ij}} = \sigma^{2}_{A_{ij}} \left(1 + \sigma^{2}_{\gamma_{i}} \right)\).
Eigenvalues of \(\textbf{S}\) are therefore enclosed within a circle
centred at \(d\) with a radius of
\(\sigma_{A_{ij}}\sqrt{SC} = \sqrt{SC\sigma^{2}_{A_{ij}}}\), while
eigenvalues of \(\textbf{A}\) have a larger radius of
\(\sqrt{SC\sigma^{2}_{A_{ij}} \left(1 + \sigma^{2}_{\gamma_{i}\right)}}\)

Note that the correlation \(\rho\) adjusts the criteria for stability as
follows\textsuperscript{\protect\hyperlink{ref-Allesina2015a}{2}},

\hypertarget{refs}{}
\leavevmode\hypertarget{ref-Tang2014b}{}%
1. Tang, S. \& Allesina, S. Reactivity and stability of large
ecosystems. \emph{Frontiers in Ecology and Evolution} \textbf{2,} 1--8
(2014).

\leavevmode\hypertarget{ref-Allesina2015a}{}%
2. Allesina, S. \& Tang, S. The stability--complexity relationship at
age 40: a random matrix perspective. \emph{Population Ecology} 63--75
(2015).
doi:\href{https://doi.org/10.1007/s10144-014-0471-0}{10.1007/s10144-014-0471-0}


\end{document}
