\documentclass[]{article}
\usepackage{lmodern}
\usepackage{setspace}
\setstretch{1}
\usepackage{amssymb,amsmath}
\usepackage{ifxetex,ifluatex}
\usepackage{fixltx2e} % provides \textsubscript
\ifnum 0\ifxetex 1\fi\ifluatex 1\fi=0 % if pdftex
  \usepackage[T1]{fontenc}
  \usepackage[utf8]{inputenc}
\else % if luatex or xelatex
  \ifxetex
    \usepackage{mathspec}
  \else
    \usepackage{fontspec}
  \fi
  \defaultfontfeatures{Ligatures=TeX,Scale=MatchLowercase}
\fi
% use upquote if available, for straight quotes in verbatim environments
\IfFileExists{upquote.sty}{\usepackage{upquote}}{}
% use microtype if available
\IfFileExists{microtype.sty}{%
\usepackage{microtype}
\UseMicrotypeSet[protrusion]{basicmath} % disable protrusion for tt fonts
}{}
\usepackage[margin=1in]{geometry}
\usepackage{hyperref}
\PassOptionsToPackage{usenames,dvipsnames}{color} % color is loaded by hyperref
\hypersetup{unicode=true,
            pdftitle={Reviewer comments and repsonses},
            pdfauthor={Brad Duthie},
            colorlinks=true,
            linkcolor=blue,
            citecolor=Blue,
            urlcolor=Blue,
            breaklinks=true}
\urlstyle{same}  % don't use monospace font for urls
\usepackage{color}
\usepackage{fancyvrb}
\newcommand{\VerbBar}{|}
\newcommand{\VERB}{\Verb[commandchars=\\\{\}]}
\DefineVerbatimEnvironment{Highlighting}{Verbatim}{commandchars=\\\{\}}
% Add ',fontsize=\small' for more characters per line
\usepackage{framed}
\definecolor{shadecolor}{RGB}{248,248,248}
\newenvironment{Shaded}{\begin{snugshade}}{\end{snugshade}}
\newcommand{\KeywordTok}[1]{\textcolor[rgb]{0.13,0.29,0.53}{\textbf{{#1}}}}
\newcommand{\DataTypeTok}[1]{\textcolor[rgb]{0.13,0.29,0.53}{{#1}}}
\newcommand{\DecValTok}[1]{\textcolor[rgb]{0.00,0.00,0.81}{{#1}}}
\newcommand{\BaseNTok}[1]{\textcolor[rgb]{0.00,0.00,0.81}{{#1}}}
\newcommand{\FloatTok}[1]{\textcolor[rgb]{0.00,0.00,0.81}{{#1}}}
\newcommand{\ConstantTok}[1]{\textcolor[rgb]{0.00,0.00,0.00}{{#1}}}
\newcommand{\CharTok}[1]{\textcolor[rgb]{0.31,0.60,0.02}{{#1}}}
\newcommand{\SpecialCharTok}[1]{\textcolor[rgb]{0.00,0.00,0.00}{{#1}}}
\newcommand{\StringTok}[1]{\textcolor[rgb]{0.31,0.60,0.02}{{#1}}}
\newcommand{\VerbatimStringTok}[1]{\textcolor[rgb]{0.31,0.60,0.02}{{#1}}}
\newcommand{\SpecialStringTok}[1]{\textcolor[rgb]{0.31,0.60,0.02}{{#1}}}
\newcommand{\ImportTok}[1]{{#1}}
\newcommand{\CommentTok}[1]{\textcolor[rgb]{0.56,0.35,0.01}{\textit{{#1}}}}
\newcommand{\DocumentationTok}[1]{\textcolor[rgb]{0.56,0.35,0.01}{\textbf{\textit{{#1}}}}}
\newcommand{\AnnotationTok}[1]{\textcolor[rgb]{0.56,0.35,0.01}{\textbf{\textit{{#1}}}}}
\newcommand{\CommentVarTok}[1]{\textcolor[rgb]{0.56,0.35,0.01}{\textbf{\textit{{#1}}}}}
\newcommand{\OtherTok}[1]{\textcolor[rgb]{0.56,0.35,0.01}{{#1}}}
\newcommand{\FunctionTok}[1]{\textcolor[rgb]{0.00,0.00,0.00}{{#1}}}
\newcommand{\VariableTok}[1]{\textcolor[rgb]{0.00,0.00,0.00}{{#1}}}
\newcommand{\ControlFlowTok}[1]{\textcolor[rgb]{0.13,0.29,0.53}{\textbf{{#1}}}}
\newcommand{\OperatorTok}[1]{\textcolor[rgb]{0.81,0.36,0.00}{\textbf{{#1}}}}
\newcommand{\BuiltInTok}[1]{{#1}}
\newcommand{\ExtensionTok}[1]{{#1}}
\newcommand{\PreprocessorTok}[1]{\textcolor[rgb]{0.56,0.35,0.01}{\textit{{#1}}}}
\newcommand{\AttributeTok}[1]{\textcolor[rgb]{0.77,0.63,0.00}{{#1}}}
\newcommand{\RegionMarkerTok}[1]{{#1}}
\newcommand{\InformationTok}[1]{\textcolor[rgb]{0.56,0.35,0.01}{\textbf{\textit{{#1}}}}}
\newcommand{\WarningTok}[1]{\textcolor[rgb]{0.56,0.35,0.01}{\textbf{\textit{{#1}}}}}
\newcommand{\AlertTok}[1]{\textcolor[rgb]{0.94,0.16,0.16}{{#1}}}
\newcommand{\ErrorTok}[1]{\textcolor[rgb]{0.64,0.00,0.00}{\textbf{{#1}}}}
\newcommand{\NormalTok}[1]{{#1}}
\usepackage{graphicx,grffile}
\makeatletter
\def\maxwidth{\ifdim\Gin@nat@width>\linewidth\linewidth\else\Gin@nat@width\fi}
\def\maxheight{\ifdim\Gin@nat@height>\textheight\textheight\else\Gin@nat@height\fi}
\makeatother
% Scale images if necessary, so that they will not overflow the page
% margins by default, and it is still possible to overwrite the defaults
% using explicit options in \includegraphics[width, height, ...]{}
\setkeys{Gin}{width=\maxwidth,height=\maxheight,keepaspectratio}
\IfFileExists{parskip.sty}{%
\usepackage{parskip}
}{% else
\setlength{\parindent}{0pt}
\setlength{\parskip}{6pt plus 2pt minus 1pt}
}
\setlength{\emergencystretch}{3em}  % prevent overfull lines
\providecommand{\tightlist}{%
  \setlength{\itemsep}{0pt}\setlength{\parskip}{0pt}}
\setcounter{secnumdepth}{0}
% Redefines (sub)paragraphs to behave more like sections
\ifx\paragraph\undefined\else
\let\oldparagraph\paragraph
\renewcommand{\paragraph}[1]{\oldparagraph{#1}\mbox{}}
\fi
\ifx\subparagraph\undefined\else
\let\oldsubparagraph\subparagraph
\renewcommand{\subparagraph}[1]{\oldsubparagraph{#1}\mbox{}}
\fi

%%% Use protect on footnotes to avoid problems with footnotes in titles
\let\rmarkdownfootnote\footnote%
\def\footnote{\protect\rmarkdownfootnote}

%%% Change title format to be more compact
\usepackage{titling}

% Create subtitle command for use in maketitle
\newcommand{\subtitle}[1]{
  \posttitle{
    \begin{center}\large#1\end{center}
    }
}

\setlength{\droptitle}{-2em}

  \title{Reviewer comments and repsonses}
    \pretitle{\vspace{\droptitle}\centering\huge}
  \posttitle{\par}
    \author{Brad Duthie}
    \preauthor{\centering\large\emph}
  \postauthor{\par}
      \predate{\centering\large\emph}
  \postdate{\par}
    \date{7 March 2019}

\usepackage{amsmath}
\usepackage{natbib}
\usepackage{lineno}
\usepackage[utf8]{inputenc}
\linenumbers
\bibliographystyle{amnatnat}

\begin{document}
\maketitle

\textbf{The following notes acknowledge and respond to reviewer comments
from a submission to \emph{PLoS Computational Biology}}.

\section{Reviewer 1}\label{reviewer-1}

\subsection{Reviewer 1 General
comments}\label{reviewer-1-general-comments}

This paper takes on a timely question: given a random matrix A and
diagonal non-negative matrix X, when will M= XA be stable. It's
important b/c this more accurately reflects the dynamics near
equilibrium in a system where species abundances are variable (and then
are equal to the diagonal elements of X). The current paper also claims
a different interpretation of the stability of a matrix of this form, in
connection with variable response rates across species.

Even though this question probably did lie dormant for many years, it
has been tackled more recently, and some of the work cited here seems to
tackle very similar questions (e.g.~Gibbs et al). But those
connections/differences aren't really explained here. In short, I'd like
to understand if what the author has identified goes beyond what is
known from other work about the effect of a diagonal matrix X on the
stability of M.

I found the presentation and framing of these results overall very
confusing. Here are some comments:

\subsection{Response to Reviewer 1 General
comments}\label{response-to-reviewer-1-general-comments}

I am grateful for the helpful comments of Reviewer 1. It is important to
first emphasise that my focal questions on the stability of complex
systems are not restricted to ecological communities (in which all rows
and columns of matrices A and M are species densities), but instead
address more general questions regarding how the stability of complex
systems is affected by varying response rate of system components (with
the exception of the specific question that I address concerning
ecological community feasibility, of course). System components might be
species densities, but they could additionally include a mix of evolving
species traits (e.g., Patel et al. 2018) or human social decisions
(e.g., management, harvesting, rewilding, land use, etc.; Hui and
Richardson 2018). Components might also not include any species, as in
the case of purely social systems (e.g., banks; Haldane \& May 2011), or
physiological or physical systems (e.g., brain networks; Gray \&
Robinson 2008; 2009). Reviewer 1 is correct that my paper offers a
different interpretation of system stability given the diagonal
non-negative matrix \(X\), one in which elements \(X_{i}\)
(\(\gamma_{i}\) in my manuscript) are interpreted as the rates at which
components respond to perturbation.

The question that I address is distinct from that of Gibbs et al (2018;
see also Reviewer 3 comments and responses). I have now revised the
manuscript to explain these differences and connections more clearly,
both mathematically and conceptually. Mathematically distinct from Gibbs
et al. (2018), I am not focused on the effect of a diagonal non-negative
matrix \(X\) on a stable matrix A as \(S \to \infty\) (i.e., as the
number of system components becomes arbitrarily large given a stable
system), and I specifically focus on a range of system complexities by
not forcing the variance of interaction strengths to scale with
increasing system size (\(\sigma = 1/\sqrt{S}\)). Conceptually distinct,
I am not focused specifically on the effect of species abundances on
ecosystem stability in a generalised Lotka-Volterra system as species
number increases. Rather, I am interested in how the stability of M is
affected by X and A at increasingly high levels of system complexity, in
which the probability of a random complex system being stable becomes
rare but non-negligible (under conditions originally identified by May
1972), and for any type of complex system. Clarifying revisions have
been now made accordingly to the Introduction and Discussion (most
notably, a new paragraph has been added to the Discussion).

\subsection{Reviewer 1 specific comment
1}\label{reviewer-1-specific-comment-1}

\begin{enumerate}
\def\labelenumi{(\alph{enumi})}
\tightlist
\item
  The most obvious interpretation (to me) for gamma\_i is as the set of
  equilibrium abundances for the set of populations, N\_i. But this
  intepretation is not explained clearly, and indeed would presumably
  not lead one to restrict the support of gamma to (0,2), or (given what
  we know about realistic distributions of abundances) to consider cases
  with half the species abundant, and half very rare.
\end{enumerate}

\subsection{Response to Reviewer 1 specific comment
1}\label{response-to-reviewer-1-specific-comment-1}

The use of random matrix theory to investigate the stability of complex
systems has perhaps most often been applied to ecological communities,
in which the complex system is restricted entirely to interacting
species where interactions are described by a square matrix A of
dimensions \(S \times S\). This matrix A can then be multiplied by a
vector describing the density of each of S species (\(N\)) to find the
change in species density (\(N'\)) over some time \(t\): \(N' = AV\)
(May 1972; 1973). It is therefore understandable why the most obvious
interpretation of a vector \(\gamma\) of length S might be equilibrium
species abundances, particularly for those with a background in
theoretical ecology; nevertheless, this is an incorrect interpretation
of \(\gamma\).

In my original manuscript, I was careful to define the vector \(\gamma\)
as the rate at which the components of a complex system (potentially,
but not necessarily, biological species) respond to system perturbation.
For the specific question of ecological community feasibility, I also
explicitly defined equilibrium abundances as the vector n. I now clarify
the use and interpretation of \(\gamma\) earlier in the manuscript, and
explicitly state that \(\gamma\) is not interpreted as a vector of
species abundances.

\subsection{Reviewer 1 specific comment
2}\label{reviewer-1-specific-comment-2}

\begin{enumerate}
\def\labelenumi{(\alph{enumi})}
\setcounter{enumi}{1}
\tightlist
\item
  If there is another interpretation of gamma\_i in terms of evolution
  of traits, then this is not adequately explained. Patel et al is cited
  in support of this, where stability of an eco-evolutionary system is
  explored using a block matrix of S species and a set of traits that
  can evolve. But it isn't clear to me in what limit the stability of
  such a system will reduce to the stability of a matrix of the form
  M=XA for community matrix A. Whereas the effect of having a
  distribution of equilibrium abundances is very clearly of this form.
\end{enumerate}

In summary I would recommend making the connection with species
equilibrium abundances clearer, and in parallel including the derivation
of gamma\_i by considering (presumably extremely fast) trait evolution.

\subsection{Response to Reviewer 1 specific comment
2}\label{response-to-reviewer-1-specific-comment-2}

There is another interpretation of \(\gamma\) that was explicitly
defined in the original manuscript. This interpretation is the rate at
which the components of a complex system (note, not just biological
species or the values of their traits) respond to system perturbation. I
have now revised my manuscript to explain this interpretation more
thoroughly, and especially more explicitly with respect to the
mathematics; \(\gamma\) modifies the matrix A to account for different
rates of response inherent to different system components.

Because \(\gamma\) is not a vector of equilibrium species abundances, I
do not make this connection as requested by Reviewer 1; instead, I have
revised my manuscript to make the interpretation of \(\gamma\) clearer,
and further emphasised that the systems of interest in my manuscript are
not restricted to species densities or biological traits.

\section{Reviewer 2}\label{reviewer-2}

\subsection{Reviewer 2 General
comments}\label{reviewer-2-general-comments}

This work it is proposes that accounting for rate variation of nodes in
responding to perturbation increases (it is incorrectly stated
``drives'') the stability in large complex systems.

The paper is well written and presented, and the results are
reproducible. However, as I will try to explain, I do not find it meets
the criteria for publication in PLOS COMPUTATIONAL BIOLOGY (i.e.,
Originality, Innovation, High importance to researchers in the field,
Significant biological and/or methodological insight, Substantial
evidence for its conclusions)

\subsection{Response to Reviewer 2 General
comments}\label{response-to-reviewer-2-general-comments}

I have changed the title from, ``Component response rate variation
drives stability in large complex systems'' to ``Component response rate
variation underlies stability in large complex systems''. I agree with
Reviewer 2 that the previous title was technically incorrect.

I am grateful for the comment on the quality of the writing and
reproducibility of my results. And while I believe that this manuscript
reports original and important work, this helpful review has helped
convinced me that it would ultimately fit better in a more
multi-disciplinary journal.

\subsection{Reviewer 2 specific comment
1}\label{reviewer-2-specific-comment-1}

\begin{enumerate}
\def\labelenumi{\arabic{enumi})}
\tightlist
\item
  The Jacobian matrix A (in the notation of the author - see pag. 8)
  already incorporates heterogeneity of the response rate, as
  \(s_i=\sum_j A_ij\) is not a priori constant, i.e.~different nodes do
  have different response rates based on their connections. Adding an
  external heterogeneity therefore, should be justified, having in mind
  a non-linear model that when linearized, gives a Jacobian that is
  \(M_j=\gamma_i A_ij\).
\end{enumerate}

\subsection{Response to Reviewer 2 specific comment
1}\label{response-to-reviewer-2-specific-comment-1}

This is an interesting and helpful comment, as it is of course true that
variation in mean row values will arise when constructing random
matrices of finite size. And this random variation could quite
reasonably be interpreted as variation in component response rates.
However, under such conditions, it is important to also emphasise that
the \emph{expected} difference between mean row values (and hence
component response rates) is still zero; i.e., differences only arise
due to error in i.i.d. random sampling for matrix element values. The
effect of this error on mean row value differences decreases with
increasing matrix size S.

In fact, we can calculate the expected standard deviation of mean row
values -- a metric of the natural variation in component response rates
-- given a randomly generated matrix, A. This value is just the standard
error of the row means. Hence for default values of \(\sigma = 0.4\) and
\(C = 1\) for, e.g., \(S = 10\), standard devation of mean row values
should be as follows:

\[SD_{mean.row.vals} = \sigma / \sqrt{S} = 0.4 / \sqrt{10} = 0.1265\]

We can do the same for small values of \(S = 2\)
(\(SD_{mean.row.vals} = 0.2828\)) and large values of \(S = 50\)
(\(SD_{mean.row.vals} = 0.0566\)). The R code below recreates this for
any \(S\); note that the -1 values on the diagonal will cause slight
deviations in practice.

\begin{Shaded}
\begin{Highlighting}[]
\CommentTok{# Reviewer 2 specific comment 1}
\NormalTok{SD_mn_row_vals <-}\StringTok{ }\NormalTok{function(}\DataTypeTok{S =} \DecValTok{10}\NormalTok{, }\DataTypeTok{iters =} \DecValTok{1000}\NormalTok{, }\DataTypeTok{sigma =} \FloatTok{0.4}\NormalTok{)\{}
    \NormalTok{sims  <-}\StringTok{ }\KeywordTok{rep}\NormalTok{(}\DataTypeTok{x =} \DecValTok{0}\NormalTok{, }\DataTypeTok{times =} \NormalTok{iters);}
    \NormalTok{for (i in }\DecValTok{1}\NormalTok{:iters)\{}
        \NormalTok{A0_dat   <-}\StringTok{ }\KeywordTok{rnorm}\NormalTok{(}\DataTypeTok{n =} \NormalTok{S *}\StringTok{ }\NormalTok{S, }\DataTypeTok{mean =} \DecValTok{0}\NormalTok{, }\DataTypeTok{sd =} \NormalTok{sigma);}
        \NormalTok{A0       <-}\StringTok{ }\KeywordTok{matrix}\NormalTok{(}\DataTypeTok{data =} \NormalTok{A0_dat, }\DataTypeTok{nrow =} \NormalTok{S); }
        \KeywordTok{diag}\NormalTok{(A0) <-}\StringTok{ }\NormalTok{-}\DecValTok{1}\NormalTok{;}
        \NormalTok{rowmns   <-}\StringTok{ }\KeywordTok{apply}\NormalTok{(}\DataTypeTok{X =} \NormalTok{A0, }\DataTypeTok{MAR =} \DecValTok{1}\NormalTok{, }\DataTypeTok{FUN =} \NormalTok{mean);}
        \NormalTok{sims[i]  <-}\StringTok{ }\KeywordTok{sd}\NormalTok{(rowmns);}
    \NormalTok{\}}
    \KeywordTok{return}\NormalTok{(}\KeywordTok{mean}\NormalTok{(sims));}
\NormalTok{\}}
\NormalTok{eg_run <-}\StringTok{ }\KeywordTok{SD_mn_row_vals}\NormalTok{(}\DataTypeTok{S =} \DecValTok{50}\NormalTok{);}
\end{Highlighting}
\end{Shaded}

Importantly, these values are small compared to the equivalent values
derived from imposing variation in component response rates a priori,
and the inclusion of \(Var(\gamma)\) adds to the variation that already
is present in \(A\). This added variation is ca 0.9523 for my
illustrative example in which half \(\gamma\) values equal 1.95 and the
other half equal 0.05. For simulations in which \(\gamma\) is randomly
sampled from 0 to 2, the value is ca 0.577.

I have edited the text of the manuscript now to make it clear that such
variation in row means of A exist prior to the inclusion of \(\gamma\),
and that such variation could reasonably be interpreted as variation in
component response rate (see Methods). I now introduce \(\gamma\) as an
a priori increase in the \emph{expected} difference between component
response rates (see Introduction) and more carefully interpret the
difference between results before and after \(\gamma\) is included.

\subsection{Reviewer 2 specific comment
2}\label{reviewer-2-specific-comment-2}

\begin{enumerate}
\def\labelenumi{\arabic{enumi})}
\setcounter{enumi}{1}
\tightlist
\item
  The latter case can be realised for a GLV, with \(\gamma_i=x^*_i\).
  This case corresponds to the case when \(\gamma_i\) and \(A\) are
  correlated, and it has been recently fully analysed {[}ref{]}.
\end{enumerate}

\subsection{Response to Reviewer 2 specific comment
2}\label{response-to-reviewer-2-specific-comment-2}

Reviewer 2 here points out the approach in which a vector affecting the
community matrix is interpreted as equilibrium species abundances
(\(x^*_{i}\)) in a generalised Lotka-Volterra ecological model, as,
e.g., in Gibbs et al (2018). In my Response to Reviewer 3 specific
comments 1 and 2, I explain how my manuscript differs from analyses of
species equilibrium abundances.

\subsection{Reviewer 2 specific comment
3}\label{reviewer-2-specific-comment-3}

\begin{enumerate}
\def\labelenumi{\arabic{enumi})}
\setcounter{enumi}{2}
\tightlist
\item
  There is no analytical analysis of the results, and yet it is known
  that stability of the linearized system dynamics only depends on the
  mean, the variance and the correlation of the matrix M (C, or S or
  sigma are not the right control parameters) {[}see Grilli et al,
  Nature Comm 2017{]}. As the variance of M increases with the
  heterogeneity in \(\gamma\), it means that when \(\gamma\) is
  different from zero, the induced correlations \(M_ij\) \(\gamma_i\)
  \(M_ki\) \(\gamma_i\) have sometimes a stabilize effect. Nevertheless,
  as it is now, this result is only shown with numerical simulations and
  there is not an analytical understanding of the result, preventing
  precise quantitative claims on the effect of the heterogeneity of
  response rates to the system stability. No mention to diagonal
  stability theory.
\end{enumerate}

\subsection{Response to Reviewer 2 specific comment
3}\label{response-to-reviewer-2-specific-comment-3}

Reviewer 2's statement that the stability of the linearised system
dynamics depends only on the mean, variance, and correlation of the
matrix M is correct specifically for the large S limit (Grilli et al.
2017; see also Gibbs et al. 2018). In other words, stability only
depends on these three values as \(S \to \infty\). Reviewer 3 also
raised this issue in their specific comments 3 and 4. In my revised
manuscript, I now clarify that my focus is specifically on the upper
range of S (not \(S \to \infty\)) at which large random systems can
reasonably be expected to be stable (see Introduction and Discussion).
For such values of S, as Reviewer 2 acknowledges (see Reviewer 2
specific comment 4 below), systems are indeed more likely to be stable
where there is heterogeneity in \(\gamma\).

Given that my focus is specifically on finite values of \(S\), I believe
that the numerical simulation approach that I have chosen is
appropriate.

\subsection{Reviewer 2 specific comment
4}\label{reviewer-2-specific-comment-4}

\begin{enumerate}
\def\labelenumi{\arabic{enumi})}
\setcounter{enumi}{3}
\tightlist
\item
  The numerical results do no lead any substantial evidence for the
  conclusion of the manuscript (``rate variation in nodes drive the
  stability of large complex systems''). In fact, by replicating the
  analysis (see attached Figures) and by the tables provided by the
  authors, it is evident that the increase of stability is at best
  around 2-3\% on the overall analysed matrix. Nevertheless, the
  unstable matrices are typically more unstable for the
  \(\gamma \neq 1\) case. Too me these results shows that the variation
  of the stability between \(\gamma_i=1\) and heterogeneous gamma\_i is
  weak.
\end{enumerate}

\subsection{Response to Reviewer 2 specific comment
4}\label{response-to-reviewer-2-specific-comment-4}

Reviewer 2 states that the numerical results both do not produce
substantial evidence for the manuscript's conclusions (sentence 1), but
also acknowledges an increase in stability (``at best around 2-3\%'';
sentence 2 -- the attached figures replicating my analysis were much
appreciated). Technically this is a contradiction, so I interpret the
reviewer to mean that, while stability does in fact increase, the actual
magnitude of this increase is not large enough to be interesting. While
I agree that the effect here is modest, I would argue that it is an
important effect to recognise when considering whether or not complex
systems are predicted to be stable, and one that has very broad
implications that might affect any complex system (i.e., not just
ecological communities).

I disagree that the unstable matrices are typically more unstable when
\(\gamma \neq 1\). I presume Reviewer 2 is interpreting ``more
unstable'' to mean that the real part of the leading eigenvalue is
higher given \(\gamma \neq 1\) for unstable systems (i.e., systems where
the real part of the leading eigenvalue is positive). While some such
values might become higher in the \(\gamma \neq 1\) condition, there is
no evidence that I can find to suggest that this is typical (i.e.,
occurs more often than not). Such a statement would seem to contradict
both the work of Gibbs (2018) and my own results (see, e.g., the change
in distribution of eigenvalues in Figure 2). Simulation also supports
the conclusion that real parts of eigenvalues in unstable matrices do
not tend to increase.

\begin{Shaded}
\begin{Highlighting}[]
\CommentTok{# Reviewer 2 specific comment 4}
\NormalTok{S       <-}\StringTok{ }\DecValTok{30}\NormalTok{;}
\NormalTok{A0_mx   <-}\StringTok{ }\OtherTok{NULL}\NormalTok{;}
\NormalTok{A1_mx   <-}\StringTok{ }\OtherTok{NULL}\NormalTok{;}
\NormalTok{iter    <-}\StringTok{ }\DecValTok{10000}\NormalTok{;}
\NormalTok{while(iter >}\StringTok{ }\DecValTok{0}\NormalTok{)\{}
    \NormalTok{r_vec    <-}\StringTok{ }\KeywordTok{rnorm}\NormalTok{(}\DataTypeTok{n =}\NormalTok{S, }\DataTypeTok{mean =} \DecValTok{0}\NormalTok{, }\DataTypeTok{sd =} \FloatTok{0.4}\NormalTok{);}
    \NormalTok{A0_dat   <-}\StringTok{ }\KeywordTok{rnorm}\NormalTok{(}\DataTypeTok{n =}\NormalTok{S *S, }\DataTypeTok{mean =} \DecValTok{0}\NormalTok{, }\DataTypeTok{sd =} \NormalTok{sigma);}
    \NormalTok{A0       <-}\StringTok{ }\KeywordTok{matrix}\NormalTok{(}\DataTypeTok{data =} \NormalTok{A0_dat, }\DataTypeTok{nrow =}\NormalTok{S, }
                       \DataTypeTok{ncol =}\NormalTok{S);}
    \NormalTok{C_dat    <-}\StringTok{ }\KeywordTok{rbinom}\NormalTok{(}\DataTypeTok{n =}\NormalTok{S *S, }\DataTypeTok{size =} \DecValTok{1}\NormalTok{, }\DataTypeTok{prob =} \NormalTok{C);}
    \NormalTok{C_mat    <-}\StringTok{ }\KeywordTok{matrix}\NormalTok{(}\DataTypeTok{data =} \NormalTok{C_dat, }\DataTypeTok{nrow =}\NormalTok{S, }
                       \DataTypeTok{ncol =}\NormalTok{S);}
    \NormalTok{A0       <-}\StringTok{ }\NormalTok{A0 *}\StringTok{ }\NormalTok{C_mat;}
    \KeywordTok{diag}\NormalTok{(A0) <-}\StringTok{ }\NormalTok{-}\DecValTok{1}\NormalTok{;}
    \NormalTok{gam1     <-}\StringTok{ }\KeywordTok{runif}\NormalTok{(}\DataTypeTok{n =}\NormalTok{S, }\DataTypeTok{min =} \DecValTok{0}\NormalTok{, }\DataTypeTok{max =} \DecValTok{2}\NormalTok{);}
    \NormalTok{A1       <-}\StringTok{ }\NormalTok{A0 *}\StringTok{ }\NormalTok{gam1;}
    \NormalTok{A0       <-}\StringTok{ }\NormalTok{A0 *}\StringTok{ }\KeywordTok{mean}\NormalTok{(gam1);}
    \NormalTok{A0_stb   <-}\StringTok{ }\KeywordTok{max}\NormalTok{(}\KeywordTok{Re}\NormalTok{(}\KeywordTok{eigen}\NormalTok{(A0)$values));}
    \NormalTok{A1_stb   <-}\StringTok{ }\KeywordTok{max}\NormalTok{(}\KeywordTok{Re}\NormalTok{(}\KeywordTok{eigen}\NormalTok{(A1)$values));}
    \NormalTok{iter     <-}\StringTok{ }\NormalTok{iter -}\StringTok{ }\DecValTok{1}\NormalTok{;}
    \NormalTok{A0_mx[iter] <-}\StringTok{ }\NormalTok{A0_stb;}
    \NormalTok{A1_mx[iter] <-}\StringTok{ }\NormalTok{A1_stb;}
\NormalTok{\}}
\NormalTok{unstable_A0 <-}\StringTok{ }\NormalTok{A0_mx[A0_mx >}\StringTok{ }\DecValTok{0}\NormalTok{];}
\NormalTok{unstable_A1 <-}\StringTok{ }\NormalTok{A1_mx[A0_mx >}\StringTok{ }\DecValTok{0}\NormalTok{];}
\end{Highlighting}
\end{Shaded}

Hence, the increase in stability I observe given \(Var(\gamma)\) is
consistent, and \(Var(\gamma)\) does not appear to make unstable systems
more unstable. Nevertheless, I do acknowledge that the evident increase
in stability is modest, and I now emphasise this better in my revised
manuscript.

\section{Reviewer 3}\label{reviewer-3}

\subsection{Reviewer 3 General
comments}\label{reviewer-3-general-comments}

Reviewer \#3: In this contribution, the author studies the variability
of component response rates (different time scales) in a general model
of population dynamics. He finds that the (asymptotic) stability of the
system is largely increased if variability in rates is introduced,
compared to the stability of a system where the vector of rates is
constant. Although the paper is well written, I have serious concerns
about its merits for publication in PLoS Computational Biology.

\subsection{Response to Reviewer 3 General
comments}\label{response-to-reviewer-3-general-comments}

I am grateful for Reviewer 3's comments. In my revised manuscript, I
further emphasise that my model is not restricted to population
dynamics, as Reviewer 3 incorrectly interpreted, but encompasses any
complex system with interacting components. Most of Reviewer 3's
additional concerns are based on a comparison with Gibbs et al. (2018;
Phys Rev E. 98:022410. doi: 10.1103/PhysRevE.98.022410. pre-print:
\url{https://arxiv.org/pdf/1708.08837.pdf}). I have revised my
manuscript to further emphasise the difference between my work and Gibbs
et al. (2018), and I address more specific concerns below.

\subsection{Reviewer 3 specific comment
1}\label{reviewer-3-specific-comment-1}

First, although the motivation of the author was (in principle)
different from that of Gibbs et al., PRE (2018), at the end there is
substantial overlap with the latter paper. Gibbs et al. conducted a
thorough study of the effect of random population abundances on the
stability of community matrices for generalized Lotka-Volterra systems,
which reduces to analyzing the stability of the matrix M=XA, where X is
a diagonal matrix with random (positive) abundances and A is a (random)
interaction matrix. This turns out to be the same problem discussed in
the present manuscript, so novelty is substantially reduced (in part
because the analysis conducted by Gibbs et al. is deeper and much more
extensive).

\subsection{Response to Reviewer 3 specific comment
1}\label{response-to-reviewer-3-specific-comment-1}

Reviewer 3 acknowledges that my motivations were different in principle
from Gibbs et al. (2018), and I have revised my manuscript to further
emphasise these differences while also acknowledging where overlap
occurs. My manuscript does not focus on primarily random population
abundances (though considers them in sections on feasibility), but
rather the reaction rates of system components to system perturbation.
While the mathematics of underlying my approach is based on analyses of
the stability of the form M = XA, the diagonal matrices are not
interpreted as abundances, and my simulations do not address the same
theoretical question as in Gibbs et al. (2018). The novelty of my
manuscript is conceptual rather than mathematical, as is also the case
for Gibbs et al. (2018). Indeed, Ahmadian et al. (2015; Phys. Rev.~E
Stat. Nonlin. Soft. Matter Phys. 91:012820. \url{doi:10.1103/PhysRevE}.
91.012820) provided a framework for investigating eigenvalue densities
of random matrices of the even more general form \(M = B + XAY\), which
would include Gibbs et al.'s (2018) M = XA. What was unique about Gibbs
et al. (2018) was its application of the properties of random matrices
of this form to a long-standing question about the relationship between
community stability and community feasibility. They showed that given an
interaction matrix (\(A\)) is stable, the vector of population
abundances (\(X\)) becomes decreasingly likely to destabilise the
resulting community matrix (\(M\)) as the community size increases.

My work applies the same mathematical framework to a different
theoretical question, and I have revised my manuscript to make this
distinction clear.

\subsection{Reviewer 3 specific comment
2}\label{reviewer-3-specific-comment-2}

Second, the main point of the present manuscript is that stability is
largely increased by introducing variability in response rates. Although
the author cites the work by Gibbs et al., is surprising that he didn't
acknowledge that his results are contradictory with those provided by
Gibbs et al. The key point of the latter paper is that population
abundances have no effect on stability, as May assumed in his seminal
paper. In fact, Gibbs et al. state that `the community matrix is stable
if an only if the interaction matrix is stable. In other words, the
abundance of species seems to not affect the sign of eigenvalues'. This
statement has to be interpreted in the limit of large systems (S going
to infinity) and in probabilistic terms, i.e., it is true almost surely.
In fact, Gibbs et al. show that the probability of matrix M=XA being
unstable given that A is stable decays exponentially with S (see Fig. 5
of Gibbs el al.) Indeed, they showed this analytically for uniformly
distributed abundances X and constant self-regulation terms, which is
precisely the case studied in the present manuscript. Conversely, the
probability of M becoming stable given that A is unstable also decays
exponentially with S (Fig. 9 of Gibbs et al.). This means that
multiplying a random matrix by a diagonal random matrix has no effect on
stability in the limit of large S.

\subsection{Response to Reviewer 3 specific comment
2}\label{response-to-reviewer-3-specific-comment-2}

Reviewer 3 has slightly mis-stated the main point of my manuscript, in
part due to some imprecision in the wording of my previous text that
Reviewer 1 noticed (particularly regarding the title, which has now been
revised). The main point of the manuscript is not that ``stability is
largely increased by introducing variability in response rates'', but
that \emph{systems in which there is variability in component response
rates are more likely to be stable}. I made this point in the second
paragraph of my Discussion:

``It is important to emphasise that variation in component response rate
is not stabilising per se; that is, adding variation in component
response rate to a particular system does not necessarily increase the
probability that the system will be stable. Rather, systems that are
observed to be stable are more likely to have varying component response
rates, and for this variation to be critical to their stability.''

I now also emphasise this point in the abstract and clarify the finite
range of S in which I am interested. Three points are particularly
important:

\begin{enumerate}
\def\labelenumi{(\arabic{enumi})}
\item
  Randomly generated matrices (\(M = XA\)) that are stable are more
  likely than not to have variation in X across the range of system
  sizes (S) that I simulated. This is (to my knowledge) a novel result
  and fully reprodible using the code underlying my simulations
  (\url{https://github.com/bradduthie/RandomMatrixStability}).
\item
  At high S, the probability that a system is stable becomes negligible;
  no such systems were found given S \textgreater{} 32. I now emphasise
  that my results apply specifically for the upper range of S
  \textgreater{} 10 for which stability is also non-negligible (i.e.,
  observed at least once in one million simulations). In this range,
  systems in which there is variation in X are indeed more stable than
  in systems in which no such variation exists. Further, this general
  pattern occurs for all simulated systems at the upper range of S,
  regardless of system connectance (C), interaction stengths
  (\(\sigma\)), or X and A element distributions (see Supplementary
  Information). In this manuscript, I am not interested in the effect of
  varying component response rates in systems that are too complex to be
  stable; my focus is entirely restricted to effects in the range where
  some stability can be reasonably expected. I explain this more
  carefully now in my Introduction and revise to state that my general
  conclusions may not apply as \(S \to \infty\).
\item
  Related to the above, I have, deliberately, not scaled interaction
  strengths (\(\sigma\)) by system size (\(S\)). My reason for doing
  this is because my manuscript is focused on the effect of component
  response rates as system complexity (\(\sigma \times \sqrt{SC}\))
  increases, not system size per se. Hence, S is increased while keeping
  all other parameter values constant so that I can evaluate the effect
  of \(Var(\gamma)\) on the stability of increasingly complex systems.
  This is fundamentally different from Gibbs et al. (2018), who scaled
  \(\sigma\) by \(S\) such that \(\sigma = 1/ \sqrt{S}\). I now state
  this explicitly in a new paragraph in the Discussion. I also show that
  my results are consistent with Gibbs et al. (2018) given the
  assumption that \(\sigma = 1/ \sqrt{S}\) in Supplementary Information.
\end{enumerate}

I hope these points adequately address Reviewer 3's concerns. There is
no contradiction between my results and those of Gibbs et al. (2018),
and I have revised my manuscript to more carefully explain why the two
works differ.

\subsection{Reviewer 3 specific comment
3}\label{reviewer-3-specific-comment-3}

This result is seemingly contradictory with the simulations reported in
the manuscript under review (Figs. 3 and 4). This is because the author
didn't scale the variance of \(A\) (\(\sigma^2\)) with \(1/S\), indeed
he maintained constant variance while increasing S. Increasing S without
scaling variances is misleading because eigenvalue distributions are not
comparable for different values of S. My feeling is that if he repeats
the figures 3 and 4 using this scaling for sigma\^{}2, then the
artifactual difference will disappear in the limit of large S, as Gibbs
et al found.

\subsection{Response to Reviewer 3 specific comment
3}\label{response-to-reviewer-3-specific-comment-3}

Reviewer 3 is of course correct that scaling the variance of A
(sigma\^{}2) with 1/S makes the difference between \(\gamma_{i} = 1\)
and \(Var(\gamma_{i})\) disappear. This was confirmed by simulation
during model development, but was not discussed in great detail in the
manuscript because it was not a focus of mine (this further highlights
the difference between my work and that of Gibbs et al. 2018). I now
explain this difference more, and I confirm Reviewer 3's feeling in the
Supplementary Information. I also explain the rational for note scaling
variances more clearly in the manuscript.

The reason for increasing values of S is to increase the total
complexity of the system (defined as \(\sigma \times \sqrt{SC}\)), and
to isolate and ultimately understand the effect of \(Var(\gamma_{i})\)
when S approaches a size at which a random system becomes too complex to
have a realistic probability of being stable (less than 1 in 1 million).
This is the reason for showing increasing system size (S) on the x-axis
of Figures 3 and 4. If I were to have scaled the variance of
off-diagonal elements of A (\(\sigma^{2}\)) with \(1/S\) as Reviewer 3
suggested (and as done in Gibbs et al. 2018), then the total complexity
of the system would not change, defeating the purpose of increasing S.
To my knowledge, there is also no a priori reason to expect systems to
have variances of interaction strengths (\(\sigma\)) that scale to their
size S, so forcing this to be the case was not justified.

Finally, while eigenvalue distributions are not \emph{identical} for
different values of S without scaling variances, it is not really
accurate to say that eigenvalue distributions are ``not comparable'';
eigenvalue distributions change with S in a predictable way that can be
compared when S is changed. It is well-established that the eigenvalues
for a random matrix M such as that used in my manuscript, and originally
by May (1972; Nature 238:413-414), will be uniformly distributed within
a circle on the complex plane within a radius of
\(\sigma\sqrt{SC} < d\), where \(-d\) is the mean of diagonal elements
of M (Tao and Tu 2010; Annals of Probability 38:2023-2065). Hence,
increasing S specifically increases the radius of this uniform
distribution, and does so predictably such that the probability that
real eigenvalues are negative (the criteria for system stability)
becomes increasingly low with increasing S. Multiplying the random
matrix A (which has the uniform distribution with a circle of radius
\(\sigma\sqrt{SC}\)) by the vector \(\gamma_{i}\) in my manuscript
resulted in an eigenvalue distribution that was more likely to be stable
at the upper range of S. I now clarify the difference between my result
and that of Gibbs et al. (2018) in this context in my revised
Discussion, and in a new section of the Supplementary Information.

\subsection{Reviewer 3 specific comment
4}\label{reviewer-3-specific-comment-4}

The author also states in this manuscript that stability can be largely
increased by `selecting' response rates so that linear stability is
forced to increase. This is an obvious technical result but (in the
light of the results by Gibbs et al.) it is also obvious that these
cases will form a set of null measure in the limit of large systems.

\subsection{Response to Reviewer 3 specific comment
4}\label{response-to-reviewer-3-specific-comment-4}

I showed that manipulating component response rates can, sometimes
greatly, increase the probability that a random system will be stable.
As with Reviewer 3 specific comment 3 (see above), I now clarify that my
focus is on the upper range of S at which large random systems can
reasonably be expected to be stable, not at the limit of
\(S \to \infty\) (as in Gibbs et al. 2018). Clarifying revisions have
been now made accordingly to the Introduction.

\section{Author References Cited}\label{author-references-cited}

Gibbs, T., Grilli, J., Rogers, T., \& Allesina, S. (2018). Effect of
population abundances on the stability of large random ecosystems.
Physical Review E, 98(2), 022410.

Gray, R. T., \& Robinson, P. A. (2009). Stability of random brain
networks with excitatory and inhibitory connections. Neurocomputing,
72(7--9), 1849--1858. \url{https://doi.org/10.1016/j.neucom.2008.06.001}

Gray, R. T., \& Robinson, P. A. (2008). Stability and synchronization of
random brain networks with a distribution of connection strengths.
Neurocomputing, 71(7--9), 1373--1387.
\url{https://doi.org/10.1016/j.neucom.2007.06.002}

Grilli, J., Adorisio, M., Suweis, S., Barabás, G., Banavar, J. R.,
Allesina, S., \& Maritan, A. (2017). Feasibility and coexistence of
large ecological communities. Nature Communications, 8.
\url{https://doi.org/10.1038/ncomms14389}

Hui, C., \& Richardson, D. M. (2018). How to invade an ecological
network. Trends in Ecology and Ecolution, xx, 1--11.
\url{https://doi.org/10.1016/j.tree.2018.11.003}

May, R. M. (1972). Will a large complex system be stable? Nature, 238,
413--414.

May, R. M. (1973). Qualitative stability in model ecosystems. Ecology,
54(3), 638--641. \url{https://doi.org/10.2307/1935352}

Patel, S., Cortez, M. H., \& Schreiber, S. J. (2018). Partitioning the
effects of eco-evolutionary feedbacks on community stability. American
Naturalist, 191(3), 1--29. \url{https://doi.org/10.1101/104505}

Haldane, A. G., \& May, R. M. (2011). Systemic risk in banking
ecosystems. Nature, 469(7330), 351--355.
\url{https://doi.org/10.1038/nature09659}


\end{document}
